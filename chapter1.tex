\chapter{Introduction}\label{cha:Introduction}
Maze has a long history spanning thousands of years. It intrigued ancient philosophers, artists, and scientists. In the modern days, we can easily say
that mazes are everywhere. From children's puzzles, traced by finger, Pac-man game and psychology experiments on mice in a laboratory to the movie Labyrinth
from 1986. But the omnipresence of mazes is even greater. Mazes also intrigued scientists who are still studying them carefully. It was soon noticed that
it may also present the maze construction as a graph. Every problem which may be presented as a graph might be considered in some ways as a maze. It opens 
an enormous variety of real-life applications of maze theory such as navigation systems, transportation route planning systems, building complexity in video
games, solving networking and electrical problems and describing complex systems in physics and chemistry. To its popularity, it can be stated with ease that 
studying the maze generating and solving algorithms, searching for difficulty measures, and searching for a new better solution for many real-life applications
is important, both for specialists and society.
\section{Motivation}
This thesis analyses algorithms which are widely used in different areas of life and technology. Modern, rapidly changing world and the drive for
new technologies pose a challenge to the commonly used solutions. The main goal of this work is to assess the possibility of creating an inference model that allows
to distinguish between different types of mazes and to decide which solving algorithm will be the best for a specific solution. The aim is to understand which parameters
have the biggest impact on the solution time. Evaluated parameters in this work are vertices degree ratio, McClendon's complexity measure, average path length and Shannon's Entropy.
Three generating algorithms were tested Binary Tree Algorithm, Aldous-Broder Algorithm and Recursive-Backtracker Algorithm. They were tested in two variants, one generic one with only one solution, 
and the second one with added cycles, directions and weights. There were also three solving algorithms tested Dijkstra's Algorithm, BFS Algorithm and $A^*$ Algorithm 
with Manhattan Distance as a heuristic function. Those algorithms were chosen because of their wide popularity in different applications but also because of their low
algorithmic complexity which allowed to generate a lot of data fast which was also important due to a lot of testing required. 

During the literature research, it was challenging to find works which are focused on formal analysis of maze complexity and maze classification other than fixed on 
McClendon's measure. Most of the research is centred on analyzing perfect mazes complexity based on McClendon's and one other maze parameters using only one solving algorithm.\cite{Kwiecien}\cite{Bellot}.
On the other hand, there is also a lot of research on maze-solving algorithms comparison and evaluating their performance, but without reference to the analysis of the 
characteristic of the problem\cite{Liu}. There is a study made by A. Karlsson \cite{Karlsson} who focuses on comparative analysis of three maze-generating algorithms
but in his work, he only uses one maze-solving algorithm. Moreover, none of those studies applies cycles, weight or direction to mazes. Therefore the fundamental idea of this work was to combine two main approaches in maze study into one. To compare and analyze different 
maze-generating algorithms and their solution time in reference to different solving algorithms. This approach offers a new input of contribution to this field of research.

\section{Research Problem}
Taking into account the findings of the studies described in the literature research part, this study's objective is to address the following research questions:\\
\begin{enumerate}
    \item [Q1.] What is the relation between the maze features, generated by Binary Tree, Aldous-Broder and Recursive-Backtracker algorithms,
     and their completion time obtained, when solved using a BFS, Dijkstra and $A^*$ algorithms?
    \item [Q2.] Which maze parameters best describe the complexity of a problem in terms of time completion?
    \item [Q3.] Which maze features are the best to distinguish different types of mazes?
\end{enumerate}
\section{Thesis Layout}
This thesis is divided into seven chapters: Introduction, Theoretical Background, Maze Real Life Related Problems, 
Maze Complexity Problems, Results, Analysis and Discussion, Model and Application Implementation and Conclusions.
Chapter 2 provides a necessary overview of maze algorithms and graph theory used in this work. Chapter 3 summarize the real-life application of algorithms 
assessed in this work. Chapter 4 describes in detail the concepts and methods of building a complexity measure for mazes. Chapter 5 presents the 
results and a detailed comparative analysis of implemented algorithms. Chapter 6 presents the results of a created classification model based on a dedicated web application.
