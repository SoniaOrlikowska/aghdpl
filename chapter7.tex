\chapter{Conclusions}\label{cha:Conclusions}
In this Master's Thesis, three maze generators and three maze solvers have been investigated and described. Chapter 2 describes in detail the theoretical foundations derived from Graph Theory,
necessary for a precise description of the essence of the studied problems. It shows the most important parameters describing graphs and mazes, and includes all definitions. In this chapter, the algorithms
used are also described in detail, along with the listings.\\
\indent Maze problems and related algorithms are very popular in real-life applications. Chapter 3 presents in detail where discussed algorithms are used, and how people benefit from them.\\ 
One of the main purposes of this work was also to provide an accurate framework for defining the maze's complexity. The most common idea behind measuring maze complexity derives only from one work. Chapter 4
proves that there are more ways to define maze complexity and that measures of complexity derived directly from Graph Theory can be used with great success.\\
\indent Chapter 5 presents the collected results and their analysis. This chapter summarizes all obtained work objectives that are consistent with the assumed objectives. The analysis of the results shows that it is possible to create a framework that allows to distinguish between different mazes based on their parameters. It is also feasible to evaluate maze complexity by more than the McClendon measure. Also, Shannon's entropy may be used as a reliable complexity measure for the studied maze generators. 
Finally, the conclusions of the results allow providing an answer that the Dijkstra solver is the best solver for evaluated algorithms, not only for perfect mazes but also for more complicated ones.\\
\indent For the purpose of this work, in order to enable others to get acquainted with the results of this work, a web application was created.  A lightweight application created in JavaScript allows seeing the results of the maze generators and solvers described in this work. In the application, it is possible to verify how the described parameters affect the appearance of the maze, their solutions and the required solution time on dedicated charts, tables and figures.\\
\indent The results and conclusions presented in this work can be used by others to expand research in this area. The framework presented here can be applied to other types of maze generators and solvers to validate conclusions for other maze problems.

