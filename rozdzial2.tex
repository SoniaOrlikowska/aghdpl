\chapter{Background}\label{cha:background}

In this chapter, I am providing a theoretical background of maze generating algorithms, maze solving algorithms, and any other theoretical concepts from the graph theory required to better understand the problems included in this paper. 

\section{Theoretical Graph Theory Background}\label{sec:theoreticalBackground}
\subsection{}
A maze can be considered as a graph, where each intersection is a vertex, and the path between them is an edge. 
\subsection{ A set}
A set is a 
\subsection{A Graph}
According to Trudeau's definition\cite{1} of a graph, it's an \textit{,,object consisting of two sets called vertex set and edge set"}. 
The vertex set is a finite, nonempty set and the edge set may be empty.A graph usually denoted as $ G = (V, E)$ is a pair of a $V$ set of nodes
 ( \textit{vertices}), and $E$ set of vertices ( \textit{edges}).\cite{2}. We can apply \textit{weight} or \textit{direction} to edges. 
 Then we can consider a graph as a \textit{ weighted graph} or \textit{directed graph}.

\subsection{A Path}
A path in a graph $G$ is a sequence of nodes $v_1, v_2,\ldots,v_k$
\section{Maze Generation Algoriths}
\section{Maze Solving Algorithms}
\subsection{Breadth-First Search Algorithm - BFS}
BFS is one of the simplest algorithms for searching a graph. As already mentioned we can consider each maze as a graph so from now on we will call 
BFS a solving algorithm or simply a solver of a given maze. From graph theory, we can state that for a given graph $ G = ( V, E) $, and distinct source 
vertex $s$, BFS explores the edges of $G$ to,,visit' each vertex directly connected with $s$. The algorithm also produces a BFS tree with $s$ root that 
contains all reachable vertexes. The,,shortest path' between $s$ and any vertex $v$ in $G$ is a simple path in the BFS tree, that is, a path containing
 the smallest number of edges. \cite{3} ( czy BFS sprawdzi się dla labiryntu o bardzo zachwianej geometrii?)



\begin{lstlisting}

\end{lstlisting}




%---------------------------------------------------------------------------
